\documentclass[12pt]{report}
\usepackage[brazil]{babel}
\usepackage[latin1]{inputenc}
\usepackage{graphicx}

\begin{document}
	\chapter{Introdu��o}
	O microprocessador, ou simplesmente CPU, � uma pe�a fundamental dos dispositivos eletr�nicos atuais. 
	Esta pe�a est� presente em computadores pessoais, tablets, smartphones e eletrodom�sticos. � respons�vel pela 			execu��o de opera��es aritm�ticas e l�gicas requisitadas pelos programas.
	
	O projeto de um microprocessador envolve circuitos grandes e complexos, � neste ponto que entra a l�gica 			program�vel. A utiliza��o desta permite escrever um c�digo que implemente a funcionalidade de um circuito 			eletr�nico. VHDL � uma das linguagens que permite a escrita deste c�digo, sendo independente de tecnologia e 			fabricante.
	
	Um Dispositivo L�gico Program�vel, ou PLD, � um hardware fixo que pode ser configurado para atender a uma 		determinada funcionalidade. A tecnologia FPGA � um exemplo de PLD, e possui in�meros chips que podem ser 		programados para executar diversas fun�oes, desde controladores de v�deo at� processadores simples.
	
	Este trabalho tem como objetivo desenvolver os blocos de um  microprocessador arquitetura x86 em VHDL e 			testar sua funcionalidade em uma FPGA.
\end{document}