%%%%%%%%%%%%%%%%%%%%%%%%%%%%%%%%%%%%%%%%%%%%%%%%%%%%%%%%%%%%%
%% HEADER
%%%%%%%%%%%%%%%%%%%%%%%%%%%%%%%%%%%%%%%%%%%%%%%%%%%%%%%%%%%%%
\documentclass[a4paper,oneside,12pt]{article}
% Alternative Options:
%	Paper Size: a4paper / a5paper / b5paper / letterpaper / legalpaper / executivepaper
% Duplex: oneside / twoside
% Base Font Size: 10pt / 11pt / 12pt


%% Language %%%%%%%%%%%%%%%%%%%%%%%%%%%%%%%%%%%%%%%%%%%%%%%%%
\usepackage[brazil]{babel} %francais, polish, spanish, ...
\usepackage[latin1]{inputenc}
\usepackage{graphicx}



%%%%%%%%%%%%%%%%%%%%%%%%%%%%%%%%%%%%%%%%%%%%%%%%%%%%%%%%%%%%%
%% DOCUMENT
%%%%%%%%%%%%%%%%%%%%%%%%%%%%%%%%%%%%%%%%%%%%%%%%%%%%%%%%%%%%%
\begin{document}

\title{Desenvolvimento de um Microprocessador 8086 RISC em VHDL \\ e Simula��o em FPGA}
\maketitle

\section{Grupo}
	D�nis Ara�jo da Silva - 18698 \\ 
	Marcos Aur�lio Freitas de Almeida Costa - 18726

\section{Orientador}
	Prof. Dr. Maur�lio Pereira Coutinho \\
	Prof. Dr. Robson Luiz Moreno - Coorientador
\section{Proposta de Trabalho}
	\quad O objetivo deste projeto � implementar um microprocessador 8086 RISC em linguagem VHDL, definindo todas as partes b�sicas, com aux�lio das ferramentas Quartus II e ModelSim-Altera. O projeto ser� validado em uma placa FPGA Cyclone II.

\section{Cronograma de Atividades}
	Segue a baixo o cronograma previsto de atividades:
	\begin{enumerate}
		\item Elabora��o da Proposta - 13/11/2013
		\item Revis�o Bibliogr�fica - 28/03/2014
		\item Workshop - 09/05/2014
		\item Entrega - 30/05/2014
		\item Defesa - Entre 16 a 18/06/2014
	\end{enumerate}
	\quad Sendo que as atividades do Trabalho Final de Gradua��o come�aram no m�s de Agosto de 2013, portanto as atividades de desenvolvimento do trabalho encontram-se em dia.
	
\section{Observa��es}
	\quad O aluno Nicolas Luiz Ribeiro Veiga - 18733, acompanhou e est� nos ajudando no desenvolvimento do trabalho, o mesmo comprometeu-se a continuar no apoio do desenvolvimento do trabalho, por�m como ele foi aceito pelo programa Ci�ncia Sem Fronteiras n�o ser� poss�vel defender o trabalho conosco. Sendo assim, ao retornar de seu interc�mbio, o aluno dar� continuidade com o trabalho, seguindo a mesma linha de racioc�nio.

\end{document}